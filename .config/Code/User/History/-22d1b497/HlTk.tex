%-------------------------------------------------------------------------------------------------%
\documentclass[12pt]{article}
%-------------------------------------------------------------------------------------------------%
\usepackage[spanish]{babel}
\usepackage[margin=1in]{geometry}
\usepackage[hidelinks]{hyperref}
\usepackage{amssymb,amsmath,amsthm,amsfonts}
\usepackage{enumerate}
\usepackage{graphicx}
\usepackage{lipsum}
\usepackage{parskip}
\usepackage{float}
\usepackage{color}
%-------------------------------------------------------------------------------------------------%
\newenvironment{solution}{\begin{proof}[Solución]}{\end{proof}}
\renewcommand{\qedsymbol}{\rule{0.7em}{0.7em}}
\newtheorem{proposition}{Proposición}
\newtheorem{observation}{Observación}
\newtheorem{afirmation}{Afirmación}
\newtheorem{definition}{Definición}
\newtheorem{corollary}{Corolario}
\newtheorem{exercise}{Ejercicio}
\newtheorem{theorem}{Teorema}
\newtheorem{example}{Ejemplo}
\newtheorem{lemma}{Lema}
\graphicspath{{Img/}}
\decimalpoint
%-------------------------------------------------------------------------------------------------%
\title{Título}
\author{Edzon Alanis}
\date{\today}
%-------------------------------------------------------------------------------------------------%
\begin{document}
%-------------------------------------------------------------------------------------------------%

\begin{center}
    \textbf{\large Problemas con Ecuaciones de Segundo Grado} \\[0.4cm]
    \today
\end{center}

\vspace*{5mm}

\begin{enumerate}
    \item Se realizará un evento musical en una escuela donde se ha determinado el
    siguiente modelo de ecuaciones: \begin{align*}
        i_{(x)} &= 70x^2+1000x+100 \\
        c_{(x)} &= 100x^2-500x-60
    \end{align*}

    Considerando los datos anteriores, calcula: \begin{enumerate}
        \item La función de utilidad. \begin{solution}
            La función de utilidad esta determinada por \[U_{(x)} = i_{(x)}- c_{(x)}\]
        \end{solution}
        \item Cuántos boletos deben vender para maximizar la ganancia del evento.
        \item Cuál es la ganancia maximizada.
    \end{enumerate}

    \item El evento musical ha sido todo un éxito, por lo que han decidido expandir su
    presentación. El director de la escuela ha notado que la compra de boletos ha
    disminuido proporcionalmente con el precio, pues si cobra \$600, asistirían 300
    personas y si cobra \$650, asistirían 200 personas. ¿Qué precio le sugerirías al
    director para maximizar los ingresos?

    La función de ingresos en función del precio se expresa de la siguiente forma:
    \[i_{(p)} = 1500p-2p^2\]

    \item Una editorial se ha dado cuenta que el libro Matemáticas para negocios es todo un
    éxito y ha decidido lanzar una segunda edición con 150 000 ejemplares a un
    precio de \$500. Sin embargo, se ha observado que cada vez que incrementan el
    precio \$10, las ventas se reducen por 250 unidades. Ayuda al dueño de la editorial
    a determinar el precio que maximice los ingresos y determina el ingreso máximo.

    Pistas: \begin{itemize}
        \item La función del precio sería: $p = 500+10n$
        \item La función del número de ejemplares sería: $q = 150000-250n$
    \end{itemize}
\end{enumerate}


%-------------------------------------------------------------------------------------------------%
\end{document}
%-------------------------------------------------------------------------------------------------%