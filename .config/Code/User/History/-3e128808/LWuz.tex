%-------------------------------------------------------------------------------------------------%
\documentclass[12pt]{article}
%-------------------------------------------------------------------------------------------------%
\usepackage[spanish]{babel}
\usepackage[margin=1in]{geometry}
\usepackage[hidelinks]{hyperref}
\usepackage{amssymb,amsmath,amsthm,amsfonts}
\usepackage{enumerate}
\usepackage{graphicx}
\usepackage{lipsum}
\usepackage{parskip}
\usepackage{float}
\usepackage{color}
%-------------------------------------------------------------------------------------------------%
\newenvironment{solution}{\begin{proof}[Solución]}{\end{proof}}
\renewcommand{\qedsymbol}{\rule{0.7em}{0.7em}}
\newtheorem{proposition}{Proposición}
\newtheorem{observation}{Observación}
\newtheorem{afirmation}{Afirmación}
\newtheorem{definition}{Definición}
\newtheorem{corollary}{Corolario}
\newtheorem{exercise}{Ejercicio}
\newtheorem{theorem}{Teorema}
\newtheorem{example}{Ejemplo}
\newtheorem{lemma}{Lema}
\graphicspath{{Img1/}}
\decimalpoint
%-------------------------------------------------------------------------------------------------%
\title{Título}
\author{Edzon Alanis}
\date{\today}
%-------------------------------------------------------------------------------------------------%
\begin{document}
\begin{center}
    \textbf{\large Funciones y sus Gráficas} \\[0.5 cm]
    \today
\end{center}
%-------------------------------------------------------------------------------------------------%
\subsection*{Analiza las funciones que se te presentan a continuación.}

\begin{enumerate}
    \item $\displaystyle{f(x) = \frac{x}{x^2-x-4}}$
    \item $\displaystyle{g(t) = \sqrt{2t-1}}$
    \item $\displaystyle{f(x) = \frac{8}{x}}$
    \item $\displaystyle{f(x) = \frac{x+1}{x^2+6x+5}}$
    \item $\displaystyle{f(x) = \frac{x}{x+8}}$
\end{enumerate}

\subsubsection*{Identifica su dominio y rango, y realiza los procedimientos correspondientes. Si es necesario agrega los gráficos o tablas que ocupaste para llegar al resultado.}

\begin{enumerate}
    \item \begin{solution}
        Para el dominio primero por ser una función racional entonces el denominador tiene que ser distinto de cero, así que el dominio serán los valores que cumplan que \[x^2-x-4 \neq 0\] entonces resolvemos la ecuación para saber los valores donde no pasa la función
        \[x^2-x-4 = 0 \Rightarrow x = \frac{-(-1) \pm \sqrt{(-1)^2-4(1)(-4)}}{2(1)} = \frac{1 \pm \sqrt{17}}{2}\]
        \[x_1 = \frac{1}{2} + \frac{\sqrt{17}}{2} \qquad x_2 = \frac{1}{2} - \frac{\sqrt{17}}{2}\]
        por lo que el dominio son todos los números reales excepto los encontrados anteriormente, lo podemos representar de la siguiente forma
        \[Dom_f = \left(-\infty, \frac{1}{2} - \frac{\sqrt{17}}{2}\right) \cup \left(\frac{1}{2} - \frac{\sqrt{17}}{2}, \frac{1}{2} + \frac{\sqrt{17}}{2}\right) \cup \left(\frac{1}{2} + \frac{\sqrt{17}}{2}, \infty\right)\]
        Para obtener el rango analizamos su grafica donde podemos observar que efectivamente se cumple el dominio que habíamos obtenido 
        \begin{figure}[H]
            \centering
            \includegraphics[width = 0.6\textwidth]{figura1.png}
            \caption{$f(x) = \frac{x}{x^2-x-4}$}
        \end{figure}
        Ya que no hay intersecciones en el eje vertical del lado positivo, negativo y el cero entonces decimos que su rango es
        \[Ran_f = (-\infty,\infty)\]
    \end{solution}
    \item \begin{solution}
        Para el dominio por ser una función radical entonces el argumento de la raíz cuadrada debe ser mayor o igual a cero, por lo que resolvemos esa desigualdad para encontrarlo
        \[2t -1 \geq 0 \Rightarrow 2t \geq 1 \Rightarrow t \geq \frac{1}{2}\] 
        esto quiere decir que el dominio serán todos los $x$ mayores e iguales a $\frac{1}{2}$, de otra forma
        \[Dom_g = \left(\frac{1}{2} , \infty\right)\]
    \end{solution}
\end{enumerate}

%-------------------------------------------------------------------------------------------------%
\end{document}
%-------------------------------------------------------------------------------------------------%