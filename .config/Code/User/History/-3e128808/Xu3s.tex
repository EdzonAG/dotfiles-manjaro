%-------------------------------------------------------------------------------------------------%
\documentclass[12pt]{article}
%-------------------------------------------------------------------------------------------------%
\usepackage[spanish]{babel}
\usepackage[margin=1in]{geometry}
\usepackage[hidelinks]{hyperref}
\usepackage{amssymb,amsmath,amsthm,amsfonts}
\usepackage{enumerate}
\usepackage{graphicx}
\usepackage{lipsum}
\usepackage{parskip}
\usepackage{float}
\usepackage{color}
%-------------------------------------------------------------------------------------------------%
\newenvironment{solution}{\begin{proof}[Solución]}{\end{proof}}
\renewcommand{\qedsymbol}{\rule{0.7em}{0.7em}}
\newtheorem{proposition}{Proposición}
\newtheorem{observation}{Observación}
\newtheorem{afirmation}{Afirmación}
\newtheorem{definition}{Definición}
\newtheorem{corollary}{Corolario}
\newtheorem{exercise}{Ejercicio}
\newtheorem{theorem}{Teorema}
\newtheorem{example}{Ejemplo}
\newtheorem{lemma}{Lema}
\graphicspath{{Img/}}
\decimalpoint
%-------------------------------------------------------------------------------------------------%
\title{Título}
\author{Edzon Alanis}
\date{\today}
%-------------------------------------------------------------------------------------------------%
\begin{document}
\begin{center}
    \textbf{\large Funciones y sus Gráficas} \\[0.5 cm]
    \today
\end{center}
%-------------------------------------------------------------------------------------------------%
\section*{Analiza las funciones que se te presentan a continuación.}

\begin{enumerate}
    \item \[f(x) = \frac{x}{x^2-x-4}\]
\end{enumerate}


%-------------------------------------------------------------------------------------------------%
\end{document}
%-------------------------------------------------------------------------------------------------%