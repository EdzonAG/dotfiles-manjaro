%-------------------------------------------------------------------------------------------------%
\documentclass[12pt]{article}
%-------------------------------------------------------------------------------------------------%
\usepackage[spanish]{babel}
\usepackage[margin=1in]{geometry}
\usepackage[hidelinks]{hyperref}
\usepackage{amssymb,amsmath,amsthm,amsfonts}
\usepackage{enumerate}
\usepackage{graphicx}
\usepackage{lipsum}
\usepackage{parskip}
\usepackage{float}
\usepackage{color}
%-------------------------------------------------------------------------------------------------%
\newenvironment{solution}{\begin{proof}[Solución]}{\end{proof}}
\renewcommand{\qedsymbol}{\rule{0.7em}{0.7em}}
\newtheorem{proposition}{Proposición}
\newtheorem{observation}{Observación}
\newtheorem{afirmation}{Afirmación}
\newtheorem{definition}{Definición}
\newtheorem{corollary}{Corolario}
\newtheorem{exercise}{Ejercicio}
\newtheorem{theorem}{Teorema}
\newtheorem{example}{Ejemplo}
\newtheorem{lemma}{Lema}
\graphicspath{{Img/}}
\decimalpoint
%-------------------------------------------------------------------------------------------------%
\title{Título}
\author{Edzon Alanis}
\date{\today}
%-------------------------------------------------------------------------------------------------%
\begin{document}
%\maketitle
%-------------------------------------------------------------------------------------------------%

\begin{center}
    \textbf{\Large Método Gauss-Jordan}\\[0.5cm]
    \today
\end{center}

\begin{enumerate}
    \item Una empresa de moda tiene disponibles 150 broches y 70 cremalleras, que desea aprovechar al 100\%, debido a una política de cero desperdicios. Por ello, se ha decidido utilizarlos en la elaboración de bolsos y pantalones de mezclilla para dama, por lo que necesitan saber cuanta mezclilla usarán y el número de piezas que pueden producir.
    
    El material a ocupar se muestra en la siguiente tabla: 
    \begin{table}[H]
        \centering
        \begin{tabular}{c|c|c|c|}
        \cline{2-4}
                                    & Metros de Mezclilla & Broches & Cremalleras \\ \hline
        \multicolumn{1}{|l|}{Bolso}    & 1.5                 & 8       & 3           \\ \hline
        \multicolumn{1}{|l|}{Pantalón} & 2                   & 2       & 1           \\ \hline
        \end{tabular}
    \end{table}

    Que se puede representar con la siguiente matriz:
    \[M = \binom{b}{p} \begin{pmatrix}
        1.5 & 8 & 3 \\
        2 & 2 & 1
    \end{pmatrix}\]

    Se observa que el número de broches que se requiere para $b$ bolsos y $p$ pantalones es de $8b+2p$ y que el número de cremalleras necesarias es de $3b+p$. El disponible de recursos es 150 y 70, por lo que se representa con el siguiente sistema de ecuaciones:
    \begin{align*}
        8b + 2p &= 150 \\
        3b + p &= 70
    \end{align*}

    Con la información, responde: 
    \vspace*{10mm}
    \begin{enumerate}
        \item ¿Cuántos bolsos y pantalones pueden producirse?
        \begin{solution}
            Resolvemos el sistema por el método de Gauss-Jordan, por lo que acomodamos el sistema en forma de matriz \[\left(\begin{array}{cc|c}  
                8 & 2 & 150 \\  
                3 & 1 & 70  \\ 
            \end{array}\right)\] Ahora resolvemos \begin{align*}
                \left(\begin{array}{cc|c}  
                    8 & 2 & 150 \\  
                    3 & 1 & 70  \\ 
                \end{array}\right)_{8R_2-3R_1 \rightarrow R_2} &\sim \left(\begin{array}{cc|c}  
                    8 & 2 & 150 \\  
                    0 & 2 & 110  \\ 
                \end{array}\right)_{R_1-R_2 \rightarrow R_1} \\
                \left(\begin{array}{cc|c}  
                    8 & 0 & 40 \\  
                    0 & 2 & 110  \\ 
                \end{array}\right)_{R_1/8, R_2/2} &\sim \left(\begin{array}{cc|c}  
                    1 & 0 & 5 \\  
                    0 & 1 & 55  \\ 
                \end{array}\right)
            \end{align*}

            Por lo tanto se requieren para la producción \[5 \; bolsos\] y \[55 \; pantalones\]
        \end{solution}
        \item ¿Cuántos metros de mezclilla se requieren para la producción? 
    \end{enumerate}

    Recuerda que la ecuación para determinarlo está dada por: 
    \[1.5b + 2p\]
    \begin{solution}
        Sustituimos en la ecuación dada y obtenemos que
        \[1.5(5)+2(55) = 117.5 \; \text{metros de mezclilla}\]
    \end{solution}
    
    \item Una panadería debe realizar un pedido de tres productos: bizcocho de fresa, pastel de chocolate y rosca de queso. Los ingredientes principales de las recetas se en listan a continuación:
    \begin{table}[H]
        \centering
        \begin{tabular}{c|c|c|c|c|}
        \cline{2-5}
                                                & \begin{tabular}[c]{@{}c@{}}Tazas \\ de Azúcar\end{tabular} & \begin{tabular}[c]{@{}c@{}}Unidades \\ de Huevo\end{tabular} & \begin{tabular}[c]{@{}c@{}}Leche \\ en Litros\end{tabular} & \begin{tabular}[c]{@{}c@{}}Número \\ de Porciones\end{tabular} \\ \hline
        \multicolumn{1}{|c|}{Bizcocho de Fresa}   & 3                                                          & 2                                                            & 1                                                          & 6                                                              \\ \hline
        \multicolumn{1}{|c|}{Pastel de Chocolate} & 2                                                          & 5                                                            & 2                                                          & 8                                                              \\ \hline
        \multicolumn{1}{|c|}{Rosca de Queso}      & 2                                                          & 4                                                            & 2                                                          & 10                                                             \\ \hline
        \end{tabular}
    \end{table}

    Actualmente, se cuenta con un stock de 15 tazas de azúcar, 24 huevos y 11 litros de leche. No hay límite en los demás ingredientes (fresas, chocolates y queso).

    Con esta información determina: ¿Cuántas porciones de cada tipo de producto se pueden realizar considerando los insumos con los que se cuenta (azúcar, huevos y leche)?

    El siguiente sistema de ecuaciones se establece con los insumos necesarios para cada producto con la disponibilidad actual en stock:
    \begin{align*}
        3b + 2p +2r &= 15 \\
        2b + 5b + 4r &= 24 \\
        1b + 2p +2r &= 11
    \end{align*}
    \begin{solution}
        Resolvemos el sistema por el método de Gauss-Jordan, para eso primero definimos el sistema como la matriz \[\left(\begin{array}{ccc|c}  
            3 & 2 & 2 & 15 \\  
            2 & 5 & 4 & 24  \\
            1 & 2 & 2 & 11 \\ 
        \end{array}\right)\]
        Resolviendo \begin{gather*}
            \left(\begin{array}{ccc|c}  
                3 & 2 & 2 & 15 \\  
                2 & 5 & 4 & 24  \\
                1 & 2 & 2 & 11 \\ 
            \end{array}\right)_{R_2-2R_3 \rightarrow R_2} \sim \left(\begin{array}{ccc|c}  
                3 & 2 & 2 & 15 \\  
                0 & 1 & 0 & 2  \\
                1 & 2 & 2 & 11 \\ 
            \end{array}\right)_{3R_3-R_1 \rightarrow R_3} \sim \left(\begin{array}{ccc|c}  
                3 & 2 & 2 & 15 \\  
                0 & 1 & 0 & 2  \\
                0 & 4 & 4 & 18 \\ 
            \end{array}\right)_{R_3-4R_2 \rightarrow R_3} \\
            \left(\begin{array}{ccc|c}  
                3 & 2 & 2 & 15 \\  
                0 & 1 & 0 & 2  \\
                0 & 0 & 4 & 10 \\ 
            \end{array}\right)_{2R_1-R_2 \rightarrow R_1} \sim \left(\begin{array}{ccc|c}  
                6 & 4 & 0 & 20 \\  
                0 & 1 & 0 & 2  \\
                0 & 0 & 4 & 10 \\ 
            \end{array}\right)_{R_1-4R_2 \rightarrow R_1} \sim \left(\begin{array}{ccc|c}  
                6 & 0 & 0 & 12 \\  
                0 & 1 & 0 & 2  \\
                0 & 0 & 4 & 10 \\ 
            \end{array}\right)_{R_1/6, R_3/4}
        \end{gather*}
    \end{solution}
\end{enumerate}



%-------------------------------------------------------------------------------------------------%
\end{document}
%-------------------------------------------------------------------------------------------------%